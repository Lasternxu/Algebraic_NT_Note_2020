\usepackage{amsmath,amssymb,amsthm} % math packages
\usepackage[usenames,svgnames,dvipsnames]{xcolor}
\usepackage{graphicx}
\usepackage{thmtools}
\usepackage[framemethod=TikZ]{mdframed}
\everymath{\displaystyle}

% section and subsection formatting
\renewcommand*{\sectionformat}{\color{RedViolet}\S\thesection\autodot\enskip}
\renewcommand*{\subsectionformat}{\color{RedViolet}\S\thesubsection\autodot\enskip}
% chapter formatting
\addtokomafont{chapterprefix}{\raggedleft}
\RedeclareSectionCommand[beforeskip=0.5em]{chapter}
\renewcommand*{\chapterformat}{%
\mbox{\scalebox{1.5}{\chapappifchapterprefix{\nobreakspace}}%
\scalebox{2.718}{\color{purple}\thechapter\autodot}\enskip}}

% Theorem Styles
% (The bluebox is for Theorems, Lemmas, Propositions, and Corollary)
% (The bluebox is for Examples)

\mdfdefinestyle{mdgreenbox}{%
	roundcorner = 10pt,
	linewidth=1pt,
	skipabove=12pt,
	innerbottommargin=9pt,
	skipbelow=2pt,
	nobreak=true,
	linecolor=ForestGreen,
	backgroundcolor=ForestGreen!5,
}
\declaretheoremstyle[
	headfont=\sffamily\bfseries\color{ForestGreen!70!black},
	mdframed={style=mdgreenbox},
	headpunct={\\[3pt]},
	postheadspace={0pt}
]{thmgreenbox}


\mdfdefinestyle{mdbluebox}{%
	roundcorner = 10pt,
	linewidth=1pt,
	skipabove=12pt,
	innerbottommargin=9pt,
	skipbelow=2pt,
	nobreak=true,
	linecolor=blue,
	backgroundcolor=TealBlue!5,
}
\declaretheoremstyle[
	headfont=\sffamily\bfseries\color{MidnightBlue},
	mdframed={style=mdbluebox},
	headpunct={\\[3pt]},
	postheadspace={0pt}
]{thmbluebox}

\declaretheorem[style=thmbluebox,name=Theorem,numberwithin=section]{theorem}
\declaretheorem[style=thmbluebox,name=Definition,sibling=theorem]{definition}
\declaretheorem[style=thmbluebox,name=Lemma,sibling=theorem]{lemma}
\declaretheorem[style=thmbluebox,name=Proposition,sibling=theorem]{proposition}
\declaretheorem[style=thmbluebox,name=Corollary,sibling=theorem]{corollary}
\declaretheorem[style=thmgreenbox,name=Problem,sibling=theorem]{problem}

\mdfdefinestyle{mdredbox}{%
	linewidth=0.5pt,
	skipabove=12pt,
	frametitleaboveskip=5pt,
	frametitlebelowskip=0pt,
	skipbelow=2pt,
	frametitlefont=\bfseries,
	innertopmargin=4pt,
	innerbottommargin=8pt,
	nobreak=true,
	linecolor=RawSienna,
	backgroundcolor=Salmon!5,
}
\declaretheoremstyle[
	headfont=\bfseries\color{RawSienna},
	mdframed={style=mdredbox},
	headpunct={\\[3pt]},
	postheadspace={0pt},
]{thmredbox}

\declaretheorem[style=thmredbox,name=Example,sibling=theorem]{example}

\mdfdefinestyle{mdblackbox}{%
	skipabove=8pt,
	linewidth=3pt,
	rightline=false,
	leftline=true,
	topline=false,
	bottomline=false,
	linecolor=black,
	backgroundcolor=RedViolet!5!gray!5,
}
\declaretheoremstyle[
	headfont=\bfseries,
	bodyfont=\normalfont\small,
	spaceabove=0pt,
	spacebelow=0pt,
	mdframed={style=mdblackbox}
]{thmblackbox}

\declaretheorem[name=Question,sibling=theorem,style=thmblackbox]{ques}
\declaretheorem[name=Remark,sibling=theorem,style=thmblackbox]{remark}


% Group Theory
\newcommand{\Gal}{\mathrm{Gal}} % Galois group
\newcommand{\Aut}{\mathrm{Aut}} % Automorphism group
\newcommand{\Sym}{\mathrm{Sym}} % Symmetric group

% Elementary Number Theory
\newcommand{\lcm}{\mathrm{lcm}} % Least common multiplier 

% Writing Style
\newcommand{\listelem}[2]{{#1}_1, {#1}_2, \cdots, {#1}_{#2}} % listing #2 elements named #1

\newcommand{\ceil}[1]{\left\lceil #1 \right\rceil}
\newcommand{\floor}[1]{\left\lfloor #1 \right\rfloor}

\newcommand{\cc}{\mathbb C}
\newcommand{\ff}{\mathbb F}
\newcommand{\nn}{\mathbb N}
\newcommand{\qq}{\mathbb Q}
\newcommand{\rr}{\mathbb R}
\newcommand{\zz}{\mathbb Z}





