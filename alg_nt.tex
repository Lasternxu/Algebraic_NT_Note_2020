\documentclass[12pt,twoside=semi,openright,numbers=noenddot]{scrbook}
\usepackage{amsmath,amssymb,amsthm} % math packages
\usepackage[usenames,svgnames,dvipsnames]{xcolor}
\usepackage{graphicx}
\usepackage{thmtools}
\usepackage[framemethod=TikZ]{mdframed}
\everymath{\displaystyle}

% section and subsection formatting
\renewcommand*{\sectionformat}{\color{RedViolet}\S\thesection\autodot\enskip}
\renewcommand*{\subsectionformat}{\color{RedViolet}\S\thesubsection\autodot\enskip}
% chapter formatting
\addtokomafont{chapterprefix}{\raggedleft}
\RedeclareSectionCommand[beforeskip=0.5em]{chapter}
\renewcommand*{\chapterformat}{%
\mbox{\scalebox{1.5}{\chapappifchapterprefix{\nobreakspace}}%
\scalebox{2.718}{\color{purple}\thechapter\autodot}\enskip}}

% Theorem Styles
% (The bluebox is for Theorems, Lemmas, Propositions, and Corollary)
% (The bluebox is for Examples)

\mdfdefinestyle{mdgreenbox}{%
	roundcorner = 10pt,
	linewidth=1pt,
	skipabove=12pt,
	innerbottommargin=9pt,
	skipbelow=2pt,
	nobreak=true,
	linecolor=ForestGreen,
	backgroundcolor=ForestGreen!5,
}
\declaretheoremstyle[
	headfont=\sffamily\bfseries\color{ForestGreen!70!black},
	mdframed={style=mdgreenbox},
	headpunct={\\[3pt]},
	postheadspace={0pt}
]{thmgreenbox}


\mdfdefinestyle{mdbluebox}{%
	roundcorner = 10pt,
	linewidth=1pt,
	skipabove=12pt,
	innerbottommargin=9pt,
	skipbelow=2pt,
	nobreak=true,
	linecolor=blue,
	backgroundcolor=TealBlue!5,
}
\declaretheoremstyle[
	headfont=\sffamily\bfseries\color{MidnightBlue},
	mdframed={style=mdbluebox},
	headpunct={\\[3pt]},
	postheadspace={0pt}
]{thmbluebox}

\declaretheorem[style=thmbluebox,name=Theorem,numberwithin=section]{theorem}
\declaretheorem[style=thmbluebox,name=Definition,sibling=theorem]{definition}
\declaretheorem[style=thmbluebox,name=Lemma,sibling=theorem]{lemma}
\declaretheorem[style=thmbluebox,name=Proposition,sibling=theorem]{proposition}
\declaretheorem[style=thmbluebox,name=Corollary,sibling=theorem]{corollary}
\declaretheorem[style=thmgreenbox,name=Problem,sibling=theorem]{problem}

\mdfdefinestyle{mdredbox}{%
	linewidth=0.5pt,
	skipabove=12pt,
	frametitleaboveskip=5pt,
	frametitlebelowskip=0pt,
	skipbelow=2pt,
	frametitlefont=\bfseries,
	innertopmargin=4pt,
	innerbottommargin=8pt,
	nobreak=true,
	linecolor=RawSienna,
	backgroundcolor=Salmon!5,
}
\declaretheoremstyle[
	headfont=\bfseries\color{RawSienna},
	mdframed={style=mdredbox},
	headpunct={\\[3pt]},
	postheadspace={0pt},
]{thmredbox}

\declaretheorem[style=thmredbox,name=Example,sibling=theorem]{example}

\mdfdefinestyle{mdblackbox}{%
	skipabove=8pt,
	linewidth=3pt,
	rightline=false,
	leftline=true,
	topline=false,
	bottomline=false,
	linecolor=black,
	backgroundcolor=RedViolet!5!gray!5,
}
\declaretheoremstyle[
	headfont=\bfseries,
	bodyfont=\normalfont\small,
	spaceabove=0pt,
	spacebelow=0pt,
	mdframed={style=mdblackbox}
]{thmblackbox}

\declaretheorem[name=Question,sibling=theorem,style=thmblackbox]{ques}
\declaretheorem[name=Remark,sibling=theorem,style=thmblackbox]{remark}


% Group Theory
\newcommand{\Gal}{\mathrm{Gal}} % Galois group
\newcommand{\Aut}{\mathrm{Aut}} % Automorphism group
\newcommand{\Sym}{\mathrm{Sym}} % Symmetric group

% Elementary Number Theory
\newcommand{\lcm}{\mathrm{lcm}} % Least common multiplier 

% Writing Style
\newcommand{\listelem}[2]{{#1}_1, {#1}_2, \cdots, {#1}_{#2}} % listing #2 elements named #1

\newcommand{\ceil}[1]{\left\lceil #1 \right\rceil}
\newcommand{\floor}[1]{\left\lfloor #1 \right\rfloor}

\newcommand{\cc}{\mathbb C}
\newcommand{\ff}{\mathbb F}
\newcommand{\nn}{\mathbb N}
\newcommand{\qq}{\mathbb Q}
\newcommand{\rr}{\mathbb R}
\newcommand{\zz}{\mathbb Z}







\title{Introduction to Algebraic Number Theory 2020 Note, Ross Program}
\author{Scott Xu}
\date{}

\begin{document}

\maketitle
\tableofcontents
\setcounter{chapter}{-1}

\chapter{Introduction}
    \noindent
    This course is hosted by professor Paul Pollack during the 2020 Ross Mathematics Program. The organization of the whole course and 
    the problems in this note is taken from his lectures, problem sets and problem sessions. \\
    \newline
    The two-week course after the usual analytic number theory sessions aims to give a short introduction on algebraic number theory, 
    and particularly, how ideas in the algebraic number theory applies to the quadratic fields. (The content of the course will be updated 
    after the course is finished as the note is updated.) \\
    \newline
    I studied the style of the \LaTeX \ note by Evan Chen's \emph{An Infinitely Large Napkin}, which is open-source on Github.

\chapter{Introduction to Algebraic Number Theory, Ross Program 2020}
\section{Lecture 1}
\begin{definition}[Algebraic Number]
    $\alpha \in \cc$ is called an algebraic number iff it's a root of some nonzero polynomial in $\qq[x]$. We denote the set of all 
    algebraic numbers as $\bar{\qq}$.
\end{definition}
\begin{definition}[Algebraic Integer]
    An algebraic integer $\alpha \in \cc$ is the root of some monic polynomial in $\zz[x]$.
    We denote the set of all algebraic integers as $\bar{\zz}$. 
\end{definition}
In summary, we have $\bar{\zz} \subset \bar{\qq} \subset \cc$. We can also check that 
$\bar{\zz} \cap \qq = \zz$. This leads us to the general discussions about number fields.\\

\begin{definition}[Number Field]
    A number field is a field of the form $\qq[\theta]$ where $\theta \in \bar{\qq}$.
\end{definition}
For example, $\zz[i], \qq[i], \qq[\omega]$ are all number fields.
\begin{definition}
    If $K$ is a number field, the ring of integers of $K$ is defined to be $\bar{\zz} \cap K$. For example, 
    the ring of integers of $\qq$ is just $\zz$.
\end{definition}
The reason why we want to study the ring of integers instead of all algebraic integers 
is that there are "too many" numbers to be studied. One observation is that $\bar{\zz}$ has no irreducible,
because for any non-unit $\alpha \in \bar{\zz}$, $\sqrt{\alpha}$ is also in $\bar{\zz}$. (Definition of units and irreducibles will
be given in Lecture 2.)

\newpage
\section{Problem Session 1}
\begin{problem}
    $\bar{\zz}$ is a ring and $\bar{\qq}$ is a field.
\end{problem}
    \begin{proof}
        We want to borrow the \textbf{Fundamental Theorem of Symmetrical Functions} here. \\
        \begin{lemma}[Fundamental Theorem of Symmetrical Functions]
            Let $R$ be a ring and let $P(\listelem{x}{n}) \in R[\listelem{x}{n}]$ be a symemtric polynomial. 
            Then $P(\listelem{x}{n}) = G(\listelem{s}{n})$ for some polynomial $G$ with coefficients from $R$.
            $s_i$ denotes the $i$-th elementary symmetrical function in $\listelem{x}{n}$, i.e., the sum of all products of $i$ terms 
            in $\listelem{x}{n}$.
        \end{lemma}
        Now we prove that $\zz$ is closed under addition and multiplication. The closure of $\qq$ can be proved in a similar way.
        Suppose that $\alpha_1$ and $\beta_1$ are roots of $P(x)$ and $Q(x)$, respectively. And suppose that $P(x) = (x-\alpha_1)(x-\alpha_2)\cdots(x-\alpha_t)$ and 
        $Q(x) = (x-\beta_1)(x-\beta_2)\cdot(x-\beta_s)$. We now find the polynomial $S(x)$ where
        $$ S(x) = \prod_{1\leq i \leq t, 1 \leq j \leq s} \left(x-(\alpha_i+\beta_j)\right) $$
        and its coefficients should lie in $\zz$ using Lemma 1.2.2 twice. Similarly we can construct
        $$T(x) = \prod_{1\leq i \leq t,1 \leq j \leq s} (x-\alpha_i \beta_j)$$
        to prove the multiplicative closure.
    \end{proof}

\begin{problem}
    It's known that the sum $1 + \sqrt{2} + \sqrt[3]{3} + \cdots + \sqrt[100]{100}$ lies strictly 
    between 111 and 112. Explain how we can quickly deduce that the sum is irrational. (From what we've already known)
\end{problem}
    \begin{proof}
        Because $\bar{\zz}$ is a ring, and each of the elements in the sum is in $\bar{\zz}$, the number is also 
        in $\bar{\zz}$. Suppose by contradiction that it's in $\qq$, then it's in $\bar{\zz} \cap \qq = \zz$, which is 
        impossible because it lies between two integers. Thus, the sum is irrational.
    \end{proof}



\begin{problem}
    Exhibit a monic polynomial in $\qq[x]$ which has $\sin(2\pi/7)$ as a root.\\
    Exhibit a monic polynomial in $\zz[x]$ which has $2\cos(2\pi/9)$ as a root.
\end{problem}

    \begin{proof}
        For the first problem, notice that 
        $$ (\cos \frac{2\pi}{7} + \sin \frac{2\pi}{7}i )^7 = e^{2\pi i} = 1$$
        $$ \sum_{j = 0}^7 \binom{7}{j} \cos^j \frac{2\pi}{7} (\sin^{7-j} \frac{2\pi}{7})i^{7-j} = 1$$
        Because both sides represent a real number, the term when $j$ is even can be treated as $0$.
        When $j$ is odd, $7-j$ is even, so we can represent $\cos^2 (2\pi/7)$ by $1-\sin^2 (2\pi/7)$.
        This generates a monic polynomial in $\qq[x]$ which has $\sin (2\pi/7)$ as a root. \\
        For the second problem, let $\zeta = e^{2\pi i/9}$ denote the ninth root of unity. Then 
        $$ 2 \cos \frac{2\pi}{9} = \zeta + \bar{\zeta} = \zeta + \zeta ^{-1}$$
        As $\zeta ^9 = 1$ and $\zeta \neq 1$, we also have
        $$ \zeta ^ 8 + \zeta ^ 7 + \cdots + \zeta + 1 = 0$$
        $$ (\zeta ^ 4 + \zeta ^{-4}) + (\zeta^3 + \zeta^{-3}) + \cdots + 1 = 0$$
        Because each $\zeta^k + \zeta^{-k}$ can be represented by a polynomial of $\zeta+ \zeta^{-1}$, we have 
        now found a monic polynomial for our $\zeta + \zeta^{-1}$, which is $2\cos (2\pi/9)$.
    \end{proof}

\begin{problem}
    Show that $\sin 1^\circ $ is an algebraic number.
\end{problem}
    \begin{proof}
        The proof is similar to Problem 1.2.4.
    \end{proof}

\begin{problem}
    Show that $\bar{\qq}$ is a fraction field of $\bar{\zz}$. That is (it remains to show that), every number in $\bar{\qq}$ can be 
    represented as the fraction of two numbers in $\bar{\zz}$.
\end{problem}
    \begin{proof}
        For any $\alpha \in \bar{\qq}$, it's a root to the polynomial $P(x) = a_n x^n + \cdots + a_1 x + a_0$ (we can let 
        $a_n = 1$ so that the polynomial is monic). 
        Let $L$ denotes the least common multiplier of the denominators of $a_i$'s written in their reduced form. 
        Now, $P(L\alpha) = L^n(P(\alpha)) = 0$ gives $Lx$ is a root of the monic polynomial $L^n P(x)$ in $\zz[x]$. Therefore, 
        $L\alpha \in \bar{\zz}$. Because $L$ is trivially in $\bar{\zz}$, we have $\alpha = L\alpha / L$ as the desired form.
    \end{proof}

\begin{problem}
    Is $1/(3+\sqrt{2})$ an algebraic integer? How to characterize the units in $\bar{\zz}$?
\end{problem}
    \begin{proof}
        For the first problem, if $\alpha = 1/(3+\sqrt{2})$ is an algebraic integer, then 
        $\bar{\alpha} = 1/(3-\sqrt{2})$ is also an algebraic integer. (This can be easily checked by the knowledge of polynomial and 
        "conjugate" roots, or viewing the map from one root to its conjugate as a ring automorphism). Because $\bar{\zz}$ is a ring, we know that $\alpha \bar{\alpha} = 1/7$ is also in $\bar{\zz}$, contradiction. \\
        For the second problem, the units in $\bar{\zz}$ can be characterize as "those $\alpha$ with minimal polynomial ending with constant term $\pm 1$". (Equivalently, the product of all its conjugates should be $\pm 1$).
    \end{proof}

\section{Lecture 2}
Recall that the study of algebraic integers focuses on $\bar{\zz} \cap K := \zz_K$, called the ring of integers of $K$, where $K = \qq[\theta]$ for some $\theta \in \bar{\qq}$.
Now we restrict our discussion on those $K$ such that $\theta$ is a root of a degree-two polynomial. 
\begin{definition}[Quadratic Number Field]
    A  quadratic number field is of the form $\qq[\theta]$ where $\theta$ is a root of a degree-two polynomial in $\qq[x]$ and $\theta \notin \qq$.
\end{definition}
If the polynomial given in the definition is $P(x) = ax^2+bx+c$, then we can express $\theta$ as $\frac{-b\pm \sqrt{b^2-4ac}}{2a}$ where $D:= b^2-4ac$ is a non-square.
Then actually $\qq[\theta] = \qq[\sqrt{D}]$. Moreover, we can uniquely determine $\qq[\sqrt{D}] = \qq[\sqrt{D'}]$ where $D'$ is not only non-square but also 
square-free. (Given as exercise)  \\
Another remark is on the convention of $\sqrt{d}$: when $d<0$, this represents $i \cdot \sqrt{|d|}$. \\
Now let $\alpha \in \qq[\sqrt{d}]$ with $d \neq 1$ being square-free. Then $\alpha = x+y\sqrt{d}$ with $x, y$ uniquely determined.
Define $\bar{\alpha} = x-y\sqrt{d}$.
\begin{remark}
    $\alpha \in \bar{\zz} \Leftrightarrow \alpha + \bar{\alpha} \in \zz, \alpha\bar{\alpha}\in \zz$.
\end{remark}
    \begin{proof}
        ($\Rightarrow$) $\alpha \in \bar{\zz}$ implies $\bar{\alpha} \in \bar{\zz}$, $\alpha + \bar{\alpha} \in \bar{\zz}$. On the other hand,
        $\alpha + \bar{\alpha} \in \qq$. Therefore, 
        $\alpha + \bar{\alpha} \in \bar{\zz}\cap \qq = \zz$. By the same argument, $\alpha\bar{\alpha} \in \zz$. \\
        ($\Leftarrow$) $\alpha$ is the root of the polynomial $P(x) = (x-\alpha)(x-\bar{\alpha})$ and $P(x) \in \zz[x]$. 
        Thus, $\alpha \in \bar{\zz}$.
    \end{proof}

\begin{remark}[Traces and Norms in quadratic field]
    $\mathrm{Tr}(\alpha) = \alpha + \bar{\alpha}$, $N(\alpha) = \alpha\bar{\alpha}$.
\end{remark}

\begin{definition}[Unit]
    An $\alpha \in \zz_K$ is called a unit if there exists $\beta \in \zz_K$ such that 
    $\alpha \beta = 1$.    
\end{definition}
\begin{definition}[Irreducible]
    A $\pi \in \zz_K$ is irreducible (an irreducible) if $\pi$ is a non-zero, non-unit element and if 
    $\pi = \alpha\beta, \alpha, \beta \in \zz_K$ implies either $\alpha$ or $\beta$ is a unit.
\end{definition}
\begin{definition}[Unique Factorization Domain, UFD]
    Let $R$ be an integral domain (a commutative ring with no zero divisors.) $R$ is a UFD if 
    $\forall \alpha \in R$, $\alpha$ being a non-zero and non-unit, $\alpha$ can be uniquely expressible 
    as $\alpha = \pi_1\pi_2\cdots\pi_k$ where the $\pi_i$'s are irreducibles.
\end{definition}
    \begin{example}[A classical example of a non-UFD]
        $\zz[\sqrt{-5}]$ is not a UFD, as $6 = 2\times 3 = (1+\sqrt{-5})(1-\sqrt{-5})$.
    \end{example}

Some exercises that characterize the quadratic field, its units and the unique factorization will be included in next section 
as exercises. \\
\newline
The problem with the non-UFD domains is that "there are not enough numbers". For example, a famous "four-number theorem"
is valid in $\zz$: if $ab = cd$ and $\gcd(a,b,c,d) = 1$, then we can write 
$a = (a,c)(a,d)$, $b=(b,c)(b,d)$, $c=(a,c)(b,c)$, $d=(a,d)(b,d)$, where $(m,n)$ denotes the 
$\gcd(m,n)$. However, the theorem is not true in 
$\zz[\sqrt{-5}]$ for the four irreducible factors in Example 1.3.7 because $\gcd(2, 1+\sqrt{-5})$ doesn't exist. \\
The way that we "restore" the unique factorization in these fields is to introduce new "numbers" that plays the role of 
"$(2, 1+\sqrt{-5})$", etc. 


\end{document} 